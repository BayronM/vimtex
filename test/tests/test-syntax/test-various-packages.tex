\documentclass{beamer}
\usepackage{subfile}
\usepackage{includegraphics}
\usepackage{pdfpages}
\usepackage{hyperref}
\usepackage{testx} " Package does not exist - test try catch block
\usepackage{biblatex}
\usepackage{csquotes}
\usepackage{cleveref}
\usepackage{varioref}
\usepackage{amsmath}
\usepackage{cases}
\usepackage{dot2texi}
\usepackage{luacode}
\usepackage{gnuplottex}
\usepackage[inline]{asymptote}
\usepackage{pythontex}

\begin{document}

\section{General}
\todostuff \todocommand \todo{content}

This text is \emph{emphasized}, \textit{italized}, \textbf{bolded}.

\def\A{A}
\def\B{\V{B}}

\newenvironment{example}[1][]{Start #1}{Stop}
\renewenvironment{example}[1][]{Start #1}{Stop}
\newenvironment{nestedexample}[1][]{%
  \renewenvironment{nested}[1][0]{%
    Outerarg #1}{%
    Innerarg ##1
  }%
  \begin{nested}{title}}{%
  \end{nested}%
}

\newcommand{\testA}[1]{Test #1}
\renewcommand{\testA}[1]{Test #1}
\newcommand{\mycommand}[1]{%
  \renewcommand{\myothercommand}[1]{%
    \dosomethingwithouterarg{#1}%
    \dosomethingwithinnerarg{##1}%
  }%
}

\section{Documentclass: beamer}
\begin{frame}[asd]
  \includegraphics[height=0.5062\linewidth]{test.png}
  \includegraphics<2>[height=0.5062\linewidth]{test.png}

  \only{asd}
  \only<2>{asd}

  \item asdasd
  \item<1-3> asdasd
\end{frame}

\section{Package: minor stuff}
% subfile
\subfile{test-syntax.tex}

% pdfpages
\includepdf{test.pdf}

\section{Package: biblatex and csquotes}
\cite{}
\citet*{}
\citep{bibtexkey1}
\citep[e.g.][]{bibtexkey2}
\citealt{}
\citealt*{}
\citealp{}
\citealp*{}
\citenum{}
\citetext{}
\citeauthor{asd}
\citeauthor*{}
\citeyear{}
\citeyearpar{}
\bibentry{}

\section{Package: cleveref}
\label[asd]{sec:cleveref}

\crefname{equation}{Eq}{Eqs}
\cref{eq:test}
\crefrange{eq:test}{eq:test2}

\section{Package: varioref}
\Vref{sec_one} does what you need. Still \vref{sec_one} works.

\section{Package: amsmath and cases}
\begin{equation}
  f(x) = 0
\end{equation}

\begin{align}
  f(x) &= \bigg\lvert 0 \bigg\rvert \\
  f(x) &= \Bigg\lVert 0 \Bigg\rVert \\
\end{align}

\begin{numcases}{y_1 = }
  f_1(x) & $x > 0$\\
  f_2(x) & $x \leqslant 0$
\end{numcases}

\begin{subnumcases}{y_1 = }
  f_1(x) & $x > 0$\\
  f_2(x) & $x \leqslant 0$
\end{subnumcases}

\section{Package: dot2texi}
\begin{dot2tex}
  graph graphname {
    a -- b;
    b -- c;
    b -- d;
    d -- a;
  }
\end{dot2tex}

\section{Package: luacode}
\directlua{
  if pdf.getminorversion() \string~= 7 then
    print "pfd version 1.7"
  end
}

\begin{luacode}
  if pdf.getminorversion() \string~= 7 then
    print "pfd version 1.7"
  end
\end{luacode}

\section{Package: gnuplottex}
\begin{gnuplot}[terminal=..., terminaloptions=...]
    plot file using 1:2 with lines
\end{gnuplot}

\section{Package: asymptote}
Here is a venn diagram produced with Asymptote, drawn to width 4cm:

\begin{asy}
  size(4cm,0);
  pen colour1=red;
  pen colour2=green;

  pair z0=(0,0);
  pair z1=(-1,0);
  pair z2=(1,0);
  real r=1.5;
  path c1=circle(z1,r);
  path c2=circle(z2,r);
  fill(c1,colour1);
  fill(c2,colour2);

  picture intersection=new picture;
  fill(intersection,c1,colour1+colour2);
  clip(intersection,c2);

  add(intersection);

  draw(c1);
  draw(c2);

  // draw("$\A$",box,z1);              // Requires [inline] package option.
  // draw(Label("$\B$","$B$"),box,z2); // Requires [inline] package option.
  draw("$A$",box,z1);
  draw("$\V{B}$",box,z2);

  pair z=(0,-2);
  real m=3;
  margin BigMargin=Margin(0,m*dot(unit(z1-z),unit(z0-z)));

  draw(Label("$A\cap B$",0),conj(z)--z0,Arrow,BigMargin);
  draw(Label("$A\cup B$",0),z--z0,Arrow,BigMargin);
  draw(z--z1,Arrow,Margin(0,m));
  draw(z--z2,Arrow,Margin(0,m));

  shipout(bbox(0.25cm));
\end{asy}

\begin{asydef}
  // Global Asymptote definitions can be put here.
  import three;
  usepackage("bm");
  texpreamble("\def\V#1{\bm{#1}}");
  // One can globally override the default toolbar settings here:
  // settings.toolbar=true;
\end{asydef}

\section{Package: pythontex}
\py{1+1}
\pyb#1+1#
\pyc@1+1@
\pys#1+1#

\pyb{print('Python says hi!')}

\begin{pycode}
  print(r'\begin{center}')
    print(r'\textit{A message from Python!}')
  print(r'\end{center}')
\end{pycode}

\begin{pyblock}
  mytext = '$1 + 1 = {0}$'.format(1 + 1)

  def myfunction(arg1, arg2):
      """docstring"""
      pass
\end{pyblock}

\begin{pycode}
print(r'\node {This is python};')
\end{pycode}

\begin{tikzpicture}
  \begin{pycode}
  print(r'\node {This is python};')
  \end{pycode}
\end{tikzpicture}

\end{document}
